\documentclass{article}
\usepackage[utf8]{inputenc}
\usepackage[english]{babel}
\usepackage{minted}
\usemintedstyle{friendly}
\makeatletter
\renewcommand\thesubsection{2.\@arabic\c@subsection}
\makeatother
\title{Chapter 1}
\author{Mahdi}
\date\today
\begin{document}

\subsection{}
We can use the original definition of make-rat to make one that handles negative rationals. Handling n or d as 0 are not asked for so those cases are ignored.
\begin{minted}{scheme}
(define (better-make-rat n d)
  (if (< (* n d) 0)
      (make-rat (* -1 (abs n)) (abs d))
      (make-rat n d)))
\end{minted}
\subsection{}
\begin{minted}{scheme}
(define (make-point x y)
  (cons x y))

(define (x-point p)
  (car p))

(define (y-point p)
  (cdr p))

(define (make-segment p1 p2)
  (cons p1 p2))

(define (start-segment s)
  (car s))

(define (end-segment s)
  (cdr s))

(define (midpoint-segment s)
  (make-point (/ (+ (x-point (start-segment s))
                    (x-point (end-segment s))) 2)
              (/ (+ (y-point (start-segment s))
                    (y-point (end-segment s))) 2)))
\end{minted}
\subsection{}
\begin{minted}{scheme}
;given points are in order starting with any point
;and going clock or counter-clock-wise
(define (make-rect p1 p2 p3 p4)
  (list p1 p2 p3 p4))

;returns the segments of the square in order
;either clock-wise or counter-clock-wise
;starting with any segment
(define (get-rect-segments r)
  (list
   (make-segment (car r) (cadr r))
   (make-segment (cadr r) (caddr r))
   (make-segment (caddr r) (cadddr r))
   (make-segment (cadddr r) (car r))))

(define (get-segment-len s)
  (sqrt (+ (expt (- (x-point (end-segment s))
                    (x-point (start-segment s))) 2)
           (expt (- (y-point (end-segment s))
                    (y-point (start-segment s))) 2))))
(define (make-rect-2 s1 s2 s3 s4)
  (list s1 s2 s3 s4))

(define (get-rect-segments-2 r) r)

(define (get-rect-param r)
  (apply + (map get-segment-len (get-rect-segments r))))

(define (get-rect-area r)
  (sqrt(apply * (map get-segment-len (get-rect-segments r)))))
\end{minted}
\subsection{}
Using the substitution model to verify the two given definitions:
\begin{minted}{scheme}(car (cons x y)) 
(car (lambda (m) (m x y))) 
((lambda (m) (m x y)) (lambda (p q) p)) 
((lambda (p q) p) x y) 
x
\end{minted}
 Without loss of generality, the following definition of cdr will work:
 \begin{minted}{scheme}
(define (car z) 
  (z (lambda (p q) q)))
\end{minted}

\subsection{}

 \begin{minted}{scheme}
 (define (cons2 a b) 
   (* (expt 2 a) (expt 3 b)))

 (define (car2 z) 
   (find-exp z 2 0))

 (define (cdr2 z) 
   (find-exp z 3 0))

 (define (find-exp n b i) 
     (if (= 0 (remainder n d)) 
         (find-exp (/ n b) b (+ x 1)) i)) 
\end{minted}
\end{document}